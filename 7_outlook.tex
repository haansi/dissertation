\chapter{Discussion and Future Work}
\label{outlook}

This thesis presented different software systems for handling mitochondrial genome data derived from various sources, covering the most prominent data formats currently used. This data is generated in high-throughput fashion by either reading the complete genome (e.g. sequencing with NGS) or by inspecting single positions on the DNA (e.g. genotyping with MicroArrays). Especially data derived from NGS devices require preprocessing, to allow the final interpretation of the data, for the detection of variants differing to a reference genome. The amount of data is growing with unprecedented speed, and requires scalable methods for data analysis. This high-throughput also requires additional quality control for sample integrity and reproducible pipelines for generation of valid result reports. Only thereafter the new data generation methods allows uncovering of new findings of origins of mutations and development in various disease. The post-processing of mtDNA data is essential, and bioinformatic tools and parameters influence the final result significantly. 
\section{Discussion}
\label{disc:sec1}
Variants in the mtDNA data, by lacking recombination (there are exception in some species like mussels) remain in linkage disequilibrium \cite{Wallace2013}. Based on this special population genetics property, mtDNA poses its own means of quality control, allowing to group mitochondrial genomes in so called haplogroups. By calculation of haplogroups based on a phylogenetic tree, this thesis describes a scalable algorithm and presents optimizations.  Thereby this tree represents an hypothesis about the evolutionary ancestry of the mitochondrial genome. 

While we developed HaploGrep in summer 2010, several tools for haplogroup classification  have been published, as presented within Chapter \ref{chapterHaplogrep}. The haplogroups are estimated by most of these tools with similar performance \cite{Bandelt2012}, with only a few tools enabling for adequate QC. With growing phylogenetic knowledge, mitochondrial haplogroups increasingly gain importance in investigating the correctness of mitochondrial sequences or genotypes, rendering phylogenetic inference indispensable \cite{Weissensteiner2016a}. The focus of this thesis is on the growing sample sizes in studies, requiring higher speed while simultaneously maintaining accuracy. 
Additional distance metrics were implemented and presented in this regard. By introducing a rule-based engine, covering a set of QCs, an easily to expand system is now implemented in HaploGrep, allowing additional QC. 

HaploGrep 2 is extended for handling data standards, like VCF files or FASTA files. In the first version, our own format HSD was proposed, which is still the most used input format, but a shift towards VCF and FASTA files already started with the publication of the updated version. Thereby the handling of FASTA files is also covered within the rule-based system, since the correct nomenclature can not always be met in an automatic manner.  Although the guidelines for mtDNA typing presented by the DNA Commission of the International Society for Forensic Genetics \cite{Parson2014} were followed, there still can exist ambiguous alignments. Therefore all alignment free derived SNPs are handled differently from SNPs derived from pairwise-alignment, and highlighted as such in the rule-based system.

The handling of mtDNA NGS data requires new methods for the large amount of data generated. Therefore we presented mtDNA-Server, which is a a web server based on Apache Hadoop, by employing the MapReduce paradigm, and splitting the data to smaller chunks to process in parallel manner. It is one tool of many to come, which allows to "democratizing bioinformatics" \footnote{http://www.nature.com/news/how-bioinformatics-tools-are-bringing-genetic-analysis-to-the-masses-1.21545}. By covering aspects from sequence alignment of FASTQ raw data to final results including heteroplasmic variants, all computational tasks in form of tools or pipelines are hidden to the end-user \cite{Weissensteiner2016b}.  Thereby we provide the mtDNA research community an easy to use web server, allowing the detection of heteroplasmic variants in a secure and reproducible way \cite{Weissensteiner2016b}. By validating the integrated heteroplasmy approach, the high sensitivity and specificity of the methods can be shown. mtDNA-Server detects heteroplasmic variants and sample contamination accurately, by analyzing artificial sample mix ups, generated in the lab.

As further shown, the haplogroup-based contamination detection in NGS-based sequencing studies, reliably detects sample mix up down to the 1 \% minor allele frequency. 
This within-sample contamination detection has significant general potential to assess data from whole exome or genome sequencing studies for potential contamination and is therefore not limited to target mtDNA sequencing. 

\section{Limitations of current work}
\label{disc:sec2}
HaploGrep 2 is based on Phylotree as its database, therefore results are highly dependent on this underlying data source \cite{Weissensteiner2016a}. As denoted in the publication, the results should not be accepted blindly. While the results from the haplogroup classification outperform manual classification, warnings and errors highlighted by the rule-based system, need manual inspection. 

As already mentioned in Chapter \ref{chapterHaplogrep}, an automatic generation of the phylogenetic tree is not available, and the manual update is performed every one to two years. This is a limitation not only for population geneticists, but also for the contamination detection, as denoted by Dickins et al. \cite{Dickins2014}. The concerns here are:
\begin{itemize}
\item if large sets of haplogroups are unknown of limited utility
\item database are needed and integrated with an analysis platform
\item implementation of search across large sample set relatively costly
\item increasing issues with interpretation in large sample size
\end{itemize}
As could be shown in the previous chapter, only the first point here is an actual limitation. The higher the haplogroup resolution, in the underlying tree, the more detailed the contamination detection can be obtained. However phylogenetic method presented in this work, as well as the method by Dickins et al. \cite{Dickins2014} are of limited utility for detecting contamination in family trios or trees with common female ancestors, independently of the haplogroup.

One further limitation of the current work derives from the fact that different reference sequences of mtDNA sequences are in use. While we strictly stick to the rCRS reference sequence, RSRS or the Yoruba reference sequence are not supported directly. mtDNA-Server converts both alternative reference sequences automatically, and annotates the variants according rCRS. While it is straight-forward from the technical view-point to rebuilt the system to work with different reference sequences, it confuses the misunderstandings about reporting mtDNA variation \cite{Bandelt2013}.

\section{Future Work}
\label{disc:sec3}
The data formats presented in Chapter \ref{chap:BioFound} are currently widely accepted, but new data formats are already in use, that requiring to support in the near future. The two most prominent are: the CRAM format and the HDF5 format. The CRAM format specification in the release 3.0\footnote{\url{https://samtools.github.io/hts-specs/CRAMv3.pdf}} describes a lossless compression of SAM files, with better compression compared to BAM and transition between the formats. Special types of the hierarchical data format (HDF5\footnote{\url{https://support.hdfgroup.org/HDF5}}) files are generated by Third-Generation Sequencing devices like the MinION Nanopore, by producing pre-base called Fast5 files. The data throughput and data quality increased significantly over the last years, yielding up to 10 Gb per run\footnote{\url{https://nanoporetech.com/about-us/news/human-genome-minion}}. The mtDNA mix-up sample with ratio 1:2 as generated for validation in Chapter \ref{chap:NGS} was already analyzed on this device, by generating single sequences of several Kb length, so that each mitochondrial molecule can now be sequenced at once.

The presented contamination detection system is available for free as service and available on GitHub\footnote{\url{https://github.com/haansi/greenVC}}. The HaploGrep source code currently in Apache Subversion (SVN) still needs to be transferred to the GIT version control system, and source code made available to the public. Currently only the executables are provided.  The user interface needs to be updated to a responsive design. The mtDNA-Server is available as service for free as well, and will be made available as Docker Image for local usage, where data is not permitted to be uploaded on our secured server.


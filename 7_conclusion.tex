\chapter{Conclusion}
\label{chap:conclusion}

\epigraph{One worthwhile task carried to a successful completion is worth a hundred half-finished tasks}{\textit{B.C. Forbes}}

In the last decade, the  field of genomics has moved to a big data science, based on the development of new sequencing devices, producing GB of of data with every single run at decreasing costs. Subsequently, the bottleneck shifted from data generation to the data analysis, requiring sophisticated pipelines, in order to derive the most important informations out of the data. Further the high-throughput of the data requires reproducible and scalable methods as well as means for quality control. 

In this thesis an information-system for a small genome, the mitochondrial DNA (mtDNA) was presented, taking these preconditions into account. Mitochondrial DNA is coding the most important bioenergetic genes, with mutations involved in a variety of diseases, tumorigenesis and ageing. mtDNA Next-Generation Sequencing (NGS) facilitates detailed insights into mtDNA, enabling to identify new mutations among thousands non-mutated (heteroplasmy) in high resolution. This higher resolution requires additional caution: phantom mutations can become apparent as false positive mutations. A further major issue is the emergence of sample contamination, observable as low level heteroplasmic mutations, leading to deceptive results in medical genetic studies.
This thesis presents our previously developed tools HaploGrep in Chapter \ref{chapterHaplogrep} and mtDNA-Server in Chapter \ref{chap:NGS}, which are merged into a contamination detection system as described in Chapter \ref{chapterContamination}. The two methods are presented, to solve the two main issues:
\begin{itemize}
\item reduce false positive mutations with the help of knowledge distilled from available data sources  
\item detect sample contaminations in mtDNA NGS data
\end{itemize}
While HaploGrep 2 automatically classifies the mtDNA profiles to haplogroups, the focus is on providing means for mtDNA Quality Control for detecting issues within those profiles. The information in the phylogenetic clusters are shown to allow controls for artificial recombination, phantom mutation and false positive and false negative mutations.
mtDNA-Server, the herein described scalable web server for mtDNA NGS data analysis, directly accepts the raw reads or mapped files without requiring deeper bioinformatic knowledge from the end-user. Thereby the data is processed reference sequence independently and annotated according to the rCRS. 
As use-case, both tools are applied to check the data quality in the 1000 Genomes Project, by highlighting common issues (e.g. contamination, numts), researchers are often not aware of. Additionally, practical post-processing guidelines for mtDNA NGS data have been developed and will be presented. Taking advantage of the mtDNA phylogeny, false positives, false negatives as well as sample contamination can be detected, thereby reducing waste and increasing the value of novel mutations discovered.

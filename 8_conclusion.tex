\chapter{Conclusion}
\label{chap:conclusion}

In the last decade, the  field of Genomics has moved to a big data science, producing GB of data at decreasing costs, based on the rapid development of new sequencing devices. Subsequently, the bottleneck shifted from data generation to the data analysis, requiring sophisticated methods, in order to derive the most important informations out of the data. Further the high-throughput of the data requires reproducible and scalable pipelines as well as means for quality control. In this thesis an information-system for a small genome, the mitochondrial DNA (mtDNA) is presented, taking the requirements for a pipeline regarding scalability, reproducibility and quality control into account. 

Mitochondrial DNA is coding the most important bioenergetic genes, with mutations involved in a variety of diseases, tumorigenesis and aging. mtDNA Next-Generation Sequencing (NGS) facilitates detailed insights into mtDNA, enabling to identify new mutations among thousands non-mutated (heteroplasmy) in high resolution. This higher resolution requires additional caution: phantom mutations can become apparent as false positive mutations. A further major issue is the emergence of sample contamination, observable as low level heteroplasmic mutations, leading to deceptive results in medical genetic studies.
This thesis presents our HaploGrep software in Chapter \ref{chapterHaplogrep} and mtDNA-Server in Chapter \ref{chap:NGS}, which are merged into a contamination detection system as described in Chapter \ref{chapterContamination}. The methods are presented, to solve the issues, as represented in Section \ref{researchaims}:
\begin{itemize}

\item User friendly and scalable information system for mtDNA data, covering common input formats for mitochondrial DNA 
\item Scalable alignment approach to run thousands of samples in feasible time, as well as a parallel approach for handling mtDNA data and detecting very rare variants (heteroplasmy) in thousands of sequence reads derived from NGS studies.
\item Scalable approach for haplogroup classification and means for quality control to reduce false positive mutations and highlighting issues with mtDNA data.
\item Detection of sample contamination in NGS studies.
\end{itemize}
While HaploGrep automatically classifies the mtDNA profiles to haplogroups with the herein presented algorithms, a further focus was on providing means for mtDNA quality control for detecting issues within those profiles, by introducing a rule-based system. The information in the phylogenetic clusters are shown to allow controls for artificial recombination, phantom mutation, false positive and false negative mutations, as presented within this work. As use case, the data set from the 1000 Genomes Project (n=2,504) was analyzed, by demonstrating the scalability and feasibility of the presented method. While new forms of mitochondrial therapies (like the mitochondrial replacement therapy \cite{Falk2016}) will become more prominent in the near future, haplogroups will become more relevant and an accurate estimation of haplogroups becomes essential. In this regard haplogroup matching is already proposed \cite{Royrvik2016}.

To be able to compare haplogroups based on the mitochondrial profile from a DNA sequence, an alignment step is required, in order to detect the single nucleotide polymorphisms on single bases, as well as insertions or deletions of larger fragments. Different approaches to perform this task are described and an implementation of an hash-index based on k-mers and dynamic programming was presented. The approach was compared to different data structures like suffix arrays and FM-Index, and implemented to run in parallel by adopting JBWA into Cloudgene, as presented in Chapter \ref{chap:NGS}. 

With the previously mentioned progress in data generation by NGS, features of the mtDNA can be researched in more depth, to learn the still poorly understood process of mutation propagation. Here a new concept of processing large amount of mtDNA data derived by massive parallel sequencing devices was presented, by taking advantage of parallel computing architectures based on the MapReduce paradigm. The herein described scalable web server for mtDNA NGS data analysis, directly accepts the raw reads or mapped files from NGS devices without requiring deeper bioinformatic knowledge from the end-user. Thereby the data is processed reference sequence independently and annotated according to the rCRS in highly parallel manner, by employing Apache Hadoop, running on the previously presented Cloudgene framework. Combined with the described maximum likelihood model as well as filter steps like strand-bias score, the workflow provides new insights in low-level mutations, as the result of different sequences per cell or tissue called heteroplasmy. As use-case, again the 1000 Genomes Phase 3 data amounting to $\sim$ 100 GB (from the small mitochondrial genome with 17Kb length) are analyzed. In order to validate the approach, several similar pipelines were compared in terms of sensitivity, specificity and precision. 

Finally the applications described within this work can all be merged into a workflow for detecting contamination in massive parallel sequencing studies. Thereby this approach is based on the concept of haplogroup detection from low-level mutations present in the sequencing data. Different approaches for contamination detection are described and the performance of the approach was evaluated based on the publicly available data-set from the 1000 Genome Consortium, providing 2,504 whole genome sequencing data and sample mix-ups in the lab. We could reliably detect the sample contamination within these data-sets, based on the presented approach.

In conclusion this work presents different algorithms, and implementations that can be applied for a specific task or combined to a system for performing quality control on mtDNA data. Taking advantage of the herein described computational pipeline based on the mtDNA phylogeny, false results as well as sample contamination can be detected. mtDNA sequences are contained in whole genome-, whole exome-  or RNA-sequencing projects and offer an inexpensive and rapid quality control as proposed as part of this work.




